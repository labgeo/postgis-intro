
% Diagramas entidad relación
% Ver ejemplo en http://heisenbugs.blogspot.com.es/2010/10/making-great-er-diagrams-without.html
\usepackage{styles/tikz-er2}
\usetikzlibrary{positioning}
\usetikzlibrary{shadows}
\usetikzlibrary{mindmap}

%% ER styles
\tikzstyle{every entity} = [top color=white, bottom color=blue!30,
draw=blue!50!black!100, drop shadow]
\tikzstyle{every weak entity} = [drop shadow={shadow xshift=.7ex,
shadow yshift=-.7ex}]
\tikzstyle{every attribute} = [top color=white, bottom color=yellow!20,
draw=yellow, node distance=7em, drop shadow]
\tikzstyle{every relationship} = [top color=white, bottom color=red!20,
draw=red!50!black!100, drop shadow]
\tikzstyle{every isa} = [top color=white, bottom color=green!20,
draw=green!50!black!100, drop shadow]


%% Árboles de directorios
%\usepackage{tikz}
%\usetikzlibrary{trees}

%\tikzstyle{every node}=[draw=black,thick,anchor=west]
%\tikzstyle{selected}=[draw=red,fill=red!30]
%\tikzstyle{optional}=[dashed,fill=gray!50]

%%%%%%%%%%%%%%%%%%%%%%%%%%%%%%%%%%%
%\begin{frame}{Modelo Jerárquico}
%\begin{columns}
%\begin{column}{0.5\textwidth}
%\begin{center}

%\begin{tikzpicture}[%
%  grow via three points={one child at (0.5,-0.7) and
%  two children at (0.5,-0.7) and (0.5,-1.4)},
%  edge from parent path={(\tikzparentnode.south) |- (\tikzchildnode.west)}]
%  \node {País}
%    child { node {Prov A}}		
%    child { node [selected] {Prov C}
%      child { node {Muni C1}}
%      child { node [selected]{Muni C2}
%      	child { node [selected] {Parc 1}}
%      	child { node [selected] {Parc2}}
%      	child { node [selected] {Parc n}}
%      }
%      child [missing] {}	
%      child [missing] {}				
%	  child [missing] {}	
%      child { node [optional] {Muni C$n$}}
%    }
%    child [missing] {}				
%    child [missing] {}	
%    child [missing] {}				
%    child [missing] {}	
%    child [missing] {}							
%    child [missing] {}				
%    child { node [optional] {Prov $n$}};
%\end{tikzpicture}
%\end{center}
%\end{column}
%
%\begin{column}{0.5\textwidth}
%El modelo jerárquico organiza los datos en una estructura ramificada (\textit{tree structure}).\\[3ex]
%
%\textbf{Consulta:}\\
%Cultivos/superficie en el municipio C2\\[3ex]
%
%\textbf{Coste:} \\
%Una búsqueda por Prov, otra por Muni de la Prov C y calculamos las estadísticas para todos. No hay que seguir buscando.
%\end{column}
%\end{columns}
%\end{frame}